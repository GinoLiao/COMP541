%!TEX root = ../iotpaper.tex

\section{Motion Authentication}
\label{sec:motionauth}

Biometrics is becoming increasingly easy to observe in commercial products. Biometric recognition is based on human physiological and/or behavioral characteristics \cite{Jain}. Fingerprint authentication, a typical example of physiological biometrics, is being applied to more and more smartphones since Apple embedded a fingerprint recognition module in iPhone. Behavioral biometrics utilizes human dynamic characteristics and has become a trend in biometrics. Some human dynamic characteristics can be used to recognize the genuineness of user correctly, such as prehension biometrics\cite{Drosou}. Similarly, motion recognition is suitable for \gls{IoT} systems which feature small sensors and low powered devices because motion recognition can achieve high accuracy with just an accelerometer \cite{RuizeXu}.

\subsection{A Possible Application}
Based on above analysis, therefore, one possible application of motion recognition is for pairing smartphone with IoT devices embedded with accelerometers. Pairing authentication is necessary because smartphone is one of the most popularly used device to control \gls{IoT} networks in today's market. We propose the following solution: the smartphone shows a specific personalized moving pattern that was defined by user previously and then the user needs to move the \gls{IoT} device in this pattern and the accelerometer data from the \gls{IoT} device is sent to the smartphone. The smartphone then compares the motion data with the existing database to check if the pattern is correct and is done by the same user with behavioral biometrics analysis. If so, the pairing succeeds. In this case, even though the smartphone is controlled by an attacker, the pairing cannot succeed. Conversely, in the opposite case, the \gls{IoT} device can ask the smartphone to perform a predefined motion and IoT device with stronger processor or connection to a server can determine if the motion data from the smartphone matches.  Also, if the \gls{IoT} is small enough, the user can hold both the \gls{IoT} device and the smartphone to perform a specific motion at the same time. Then both devices can just compare the motion data from the other device with its own. In this case, no database is needed.

\subsection{Feasibility}

As mentioned above, motion recognition with only accelerometer gives high accuracy, thus how to set up a biometric database to check if the motion is performed by correct user become the main concern of this solution. A feasible solution provided by Guerra et al. \cite{Casanova} is as the following. They let the user to invent their personal gesture and offer 7 samples to the database. Users give really different gestures, such as writing a word and drawing some symbol in the air. All these gestures are collect while holding a device with an accelerometers embedded. The result from Guerra et al. \cite{Casanova} shows that the impostors has a very low chance of imitating successfully. This method can be applied to the pairing between smartphone and \gls{IoT} devices. Smartphone can just show a hint of the gesture or even nothing to have a higher security level.

\subsection{Equipment and Algorithm}
Obviously, a smartphone and an additional device embedded with accelerometers that can connect with the smartphone in bluetooth or some other ways are needed for this solution. Other than these, an efficient recognition algorithm: uWave can be used in this solution, because this algorithm requires only one accelerometer and is proved by Jiayang et al. \cite{Liu:2009, LiuuWave} from Rice University that it well recognized personalized gestures.