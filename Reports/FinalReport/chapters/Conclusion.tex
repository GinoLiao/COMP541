%!TEX root = ../iotpaper.tex

\section{Conclusion \& Future Work}
\label{sec:Conclusion}

%Summarize contributions/key results 
Authentication without a central authority can be tricky, but biometrics can be one valuable factor for authentication without that resource. In this paper, we have shown that authentication with gestures has the potential to validate a user. In the Gesture Library model, we show that it is difficult to imitate most gestures and set a simple threshold for authentication based on a previously calibrated library. Although we are not able to reduce both false positives and false negatives to less than 10\%, we can control the equilibrium point by changing the number of challenges by the verifier. In the Simultaneous Gestures model, we have shown that copying a gesture visually---even with prior knowledge of what the gesture will be---is difficult, and the distance is typically one order of magnitude larger than the distance of the legitimate user if he holds the prover and verifier together. 

In the future, we recommend dynamically updating calibration data with new authentications so that the calibration data does not become stagnant. A threshold algorithm also needs to be set for the Simultaneous Gestures model, potentially one that extracts information about the time and complexity of the gesture itself.

%Future work
%-Dynamically update calibration based on historical information.
%-Threshold setting for the simultaneous gestures case (involves time and complexity)