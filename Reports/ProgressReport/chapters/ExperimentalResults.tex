%!TEX root = ../iotpaper.tex

\section{Experimental Results}
\label{sec:Results}


\subsection{Gesture Recognition on a Single Device}
At first, we want to test our assumption that there are enough difference between time samples of the same gesture performed by different users. So all three team members performs the same gesture with the same iPhone device.

The current data includes 10 samples of writing letter `e' by Gino, 1 sample of writing letter `s', `e' by all three members. All accelerometer data are stores as `.mat' file and each one comes with a plot of acceleration on `x', `y', `z' axes versus time. The time length of each data file depends on how long the user finishes the gesture. The start time and end time of capturing data depend on user.

Then, we performs uWave algorithm. We first quantized the acceleration data, calculated the distance between two time samples and do dynamic time wraping to check if two gestures match. In our experiment, Gino is the original user and Joe and Henry are the attackers. Table~\ref{table:distanceOnSingleDevice} shows our result of distances to standard deviation of three users.

\begin{table}

\begin{center}
  \begin{tabular}{ c | c | c}
    \hline
    \backslashbox{Pairer}{Gesture}
      & Letter `e' & Letter `s' \\ \hline
    Gino & 5 & 6 \\ \hline
    Joe & 8 & 9 \\	\hline
    Henry & 8 & 9 \\
    \hline
  \end{tabular}
\end{center}
\caption{Distance to calibrated original time sample} % title of Table
\label{table:distanceOnSingleDevice}
\end{table}


 

