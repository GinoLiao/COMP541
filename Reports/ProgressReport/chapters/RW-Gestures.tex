%!TEX root = ../iotpaper.tex

\subsection{Gesture \& Motion Recognition on a Single Device}

Several efficient gesture-recognition algorithms already exist. Jiayang et al. \cite{Liu:2009, LiuuWave} presented an algorithm called uWave that is based on a single accelerometer. uWave quantizes the acceleration data to reduce computational load and uses dynamic time warping to measure similarities between two time series of accelerometer data. Template adaptation deals with gesture variation over the time. Ahmad and Shahrokh \cite{Ahmad:2010} also proposes a gesture recognition system that uses only one 3-axis accelerometer. The system temporally compresses the acceleration time series to filter out variations not intrinsic to the gesture itself and reduces the size of the acceleration signals for next step of dynamic time warping. Then, the system uses affinity propagation to find a good set of exemplars from all data points. Finally, they implemented compressive sensing to recognize a repetition of a gesture. 

For the best user experience, gesture recognition should be in real-time and easy to use. Instead of using a button to indicate start time and end time of a gesture motion, Zoltan \cite{Zoltan} proposes an automatic segmentation method and uses two classification algorithms: Hidden Markov Models and Support Vector Machine to to give high accuracy. This system has great performance and low response time, thus a good model for \gls{IoT} devices.
