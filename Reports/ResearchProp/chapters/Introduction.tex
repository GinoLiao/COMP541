%!TEX root = ../iotpaper.tex

\section{Introduction}
\label{sec:Introduction}

Network security faces several new challenges in an \gls{IoT} scenario. First, \gls{IoT} devices are typically wireless sensors or low-powered appliances, meaning they are resource-constrained. This makes the traditional cryptographic defense that support confidentiality, integrity, and authencity difficult to implement \cite{authmodels}. Second, \gls{IoT} devices are often devices such as health sensors and home network appliances. These devices by nature carry personal and dynamic information about the user that can be leaked if proper security measures are not implemented \cite{Jing}. Third, \gls{RFID} technology uses passive tags to help a user identify an object or location. However, this technology is vulnerable to both non-malicious collisions and adversarial denial-of-service attacks \cite{RFID}. 

For the remainder of this paper, we propose three potential semester projects and discuss existing works in each of these fields. In \autoref{sec:crypto}, we aim to cross-compare multiple existing lightweight cryptographic algorithms on the same platform. In \autoref{sec:motionauth}, we propose using gesture recognition on multiple devices for pairing authentication. Finally, in \autoref{sec:otherauth}, we implement and evaluate a reputation system for \gls{IoT} networks under multiple environments. 